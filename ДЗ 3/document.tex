\documentclass[12pt]{urticle} 
\usepackage[left=20mm, top=16mm, right=16mm, bottom=20mm]{geometry} 
%\usepackage{graphicx}
%\usepackage{wrapfig}
\graphicspath{{../pictures/}}
%\DeclareGraphicsExtensions{.pdf,.png,.jpg,.eps}
\usepackage{cmap}					% поиск в PDF
\usepackage{mathtext} 				% русские буквы в формулах
\usepackage[T2A]{fontenc}	
\usepackage[utf8x]{inputenc} 
%\usepackage[russian]{babel} 
\usepackage{amsmath,amsfonts,amssymb,amsthm,mathtools} 
\usepackage{icomma} % "Умная" запятая: $0,2$ --- число, $0, 2$ --- перечисление
\usepackage{euscript}	 % Шрифт Евклид
\usepackage{mathrsfs} % Красивый матшрифт
\usepackage{indentfirst}     % Отступ в первом абзаце
%% Перенос знаков в формулах (по Львовскому)
\newcommand*{\hm}[1]{#1\nobreak\discretionary{}
	{\hbox{$\mathsurround=0pt #1$}}{}}

%% Свои команды
%DeclareMathOperator{\sgn}{\mathop{sgn}}
\newcommand{\te}{\ensuremath{\Rightarrow}}
\newcommand{\y}{\ensuremath{\angle}}
\newcommand{\ABC}{\ensuremath{\triangle ABC\,}}
\newcommand{\tr}{\ensuremath{\triangle}}
\newcommand{\ca}{\ensuremath{\cos\alpha}}
\newcommand{\sa}{\ensuremath{\sin\alpha}}
\newcommand{\cb}{\ensuremath{\cos\beta}}
\newcommand{\sib}{\ensuremath{\sin\beta}}
\newcommand{\ov}{\ensuremath{\overline}}
\newcommand{\x}{\cdot}
\newcommand{\st}{\ensuremath{\longrightarrow}}
%DeclareMathOperator{\Sum}{\mathop{Sum}}
\DeclareMathOperator{\Sum}{\mathop{Sum}}

\begin{document}
	
\begin{titlepage}		
\begin{center}
\large 	Московский физико-технический университет \\
Факультет общей и прикладной физики \\
\vspace{0.2cm}
Учебная программа\\
"<Квантовая теория поля, теория струн и математическая физика">

\vspace{4.5cm}
II семестр 2016-2017 учебного года \\ \vspace{0.1cm}
\large Домашнее задание №3: \\ \vspace{0.1cm}
\LARGE \textbf{Взаимодействия элементарных частиц и их свойства}
\end{center}
\vspace{2.3cm} \large

\begin{center}
		 Автор: \\
 Иванов Кирилл,
 625 группа
\vspace{10mm}


\end{center}

\begin{center} \vspace{50mm}
г. Долгопрудный \\ 
20 апреля 2017 года
\end{center}
\end{titlepage}

\section{Задача №1}

\begin{equation}
 p + p \st p + p + p + \ov{p}
\end{equation}

Введём инвариантную массу:
\begin{equation}
M_{inv}^2  = \left(\sum E_i\right)^2 - \left(\sum p_i\right)^2
\end{equation}

Обозначим энергию налетающего протона за $ E  $,  его массу за $ m $ и импульс за $ p $, а аналогичные параметры частиц после реакции за $ E_i, p_i $ соответственно, $ i = 1, 2, 3, 4. $ Энергию покоя протона обозначим $ E_0 = mc^2 $.
Принимая скорость света $ c = 1 $, запишем инвариантную массу в лабораторной системе отсчёта:

\begin{equation}
M_{inv_{до}}^2 = \left(E + E_0 \right)^2 -  p^2 = M_{inv_{после}}^2  = \left(\sum\limits_{i=1}^4 E_i\right)^2 - \left(\sum\limits_{i=1}^4  p_i\right)^2
\end{equation} 
Так как мы ищем минимальную энергию, то возьмём $ \left(\sum_{i=1}^4  p_i\right)^2 = 0  \te M_{inv_{после}}^2  = \left(\sum_{i=1}^4 E_i\right)^2 \hm{=} 16m^2. $
Воспользуемся формулой энергии релятивистской частицы $ E = \sqrt{p^2 + m^2} $:
$$ M_{inv_{до}}^2 = E^2 + 2mE + m^2 -  p^2 = p^2 + m^2 + 2mE + m^2 -  p^2  = 2mE + 2m^2;$$
$$ 2mE_{min} + 2m^2 = 16m^2 \te E_{min} = 7m. $$

Зная массу протона $ m = 940 \text{MeV}/c^2, $  находим $ E \approx 6,6  \,  \text{GeV}$.
\begin{center}
	{\fbox{Ответ: $ E_{min} \approx 6,6  \, \text{GeV} $}} \\
\end{center} 

\section{Задача №2}

Для заряженного пиона возможен распад на электрон и электронное антинейтрино (и их "<антикомбинацию">):

\begin{equation}
\left\{
\begin{aligned}
&\pi^- \st e^- + \overline{\nu_e}\\
&\pi^+ \st e^+ + \nu_e \\
\end{aligned}
\right.
\end{equation}

Действительно, здесь выполняется как закон сохранения знакового числа, так и закон сохранения лептонного числа (т.к. у пи-мезона $ L = 0 $), а у электрона и электронного антинейтрино (и их "<антикомбинации">) лептонные числа противоположны.

Возможно также превращение нейтрально заряженного пиона в два гамма-кванта:

\begin{equation}
\pi^0 \st 2\gamma
\end{equation}


\section{Задача №3} \par
Эти адроны разделены на две логические группы. \\


1) Барионы состоят из 3 кварков: 

\begin{tabular}{|c|c|c|c|c|c|c|c|c|c|c|c|c|}
	\hline 
	Барионы&$  p $ & $ n $ &$  \Lambda $ & $ \Sigma^+ $ &$  \Sigma^- $ & $ \Sigma^0 $ & $ \Xi^+ $ & $ \Xi^- $ & $ \Xi^0 $    \\ 
	\hline 
	Состав & $ uud $ & $ ddu $ & $ uds $ & $ uus $ & $ dds $ & $ uds $ & $ usc $ & $ dss $ & $ uss $   \\ 
	\hline 
\end{tabular} 
\\

2) Вторая логическая группа --- мезоны, состоящие из 2 кварков: 
\par
\vspace{0.1cm}
\begin{tabular}{|c|c|c|c|c|c|c|c|}
	\hline 
	Мезоны &$  \pi^+ $ & $ \pi^-  $& $ \pi^0 $ &$  K^+ $ & $ K^-  $& $ K^0 $ & $ \eta $ \\ 
	\hline 
	Состав &$  u\ov{d} $ & $ u\ov{d} $ & $ \dfrac{u\ov{u} - d\ov{d}}{\sqrt{2}} $ & $ u\ov{s} $ & $ u\ov{s}  $& $ \dfrac{d\ov{s} - s\ov{d}}{\sqrt{2}}  $ &  $ \dfrac{u\ov{u} + d\ov{d} - s\ov{s}}{\sqrt{6}} $  \\ 
	\hline 
\end{tabular} 

Здесь чертой сверху обозначен антикварк.

\section{Задача №4}

У нас произошло превращение: 
\begin{equation}
s \st u + W^-
\label{us}
\end{equation}

При этом возможно дальнейший распад бозона $ W^- $ на электрон и электронное антинейтрино. 

$ \Sigma^- $ состоит из кварков $  dds $, а нейтрон --- из $ ddu $.  Рассмотрим нашу реакцию: 

$$ \Sigma^- \st n + e^- + \ov{\nu_e} $$. 

Из вышесказанного нетрудно заметить, что на более глубинном уровне проиходит превращение $ s$-кварка сигма-гипериона в $ u $-кварк нейтрона согласно \eqref{us}, а затем происходит распад бозона $ W^- $, т.е. 
\begin{equation}
\eqref{us} \Leftrightarrow dds \st ddu + W^- \st ddu + e^- + \ov{\nu_e}
\end{equation}

Поэтому такая реакция возможна.

Напротив, $ \Sigma^+ $ состоит из $ uus $ кварков, и если $ s $ превращается в $ u $, то $ uu $ не превращается в $ dd $, поэтому данная реакция невозможна, хотя $ W^+ $-бозон и может превратиться в позитрон и электронное нейтрино.

\section{Задача №5}

\textbf{Фермионы} --- элементарные частицы с \textit{полуцелым} спином, а\textbf{ бозоны} --- с \textit{целым}.

\begin{itemize}
	
	\item У кварков спин равен $ \frac12\hbar \te $  \textbf{$ d $-частица --- фермион.}
	
	\item Атом водорода состоит из просто-напросто одного протона, спин которого  $ \frac12\hbar \te $ \textbf{он тоже фермион}.
	
	\item Если $ N = Z $, т.е. в ядре одинаковое число нейтронов и протонов, то ядро будет обладать нулевым спином $ \te $ \textbf{оно будет бозоном.} Однако в обратном случае ($ N\neq Z$) спин будет ненулевым, но неизвестно, целым или нецелым (по-разному в конкретных случаях). \textbf{Тогда ядро может является как бозоном, так и фермионом.}
	
	\item Атом водорода $ H_2 $ состоит из двух фермионов со спином  $ \frac12\hbar  $. По правилу сложения спинов такая молекула может обладать как нулевым, так и единичным спинами (такие молекулы называются \textit{параводород} и \textit{оптоводород} соответственно). Получаем, что в обоих случаях такие молекулы являются \textbf{бозонами.}
		
\end{itemize}

\section{Задача №6}

\begin{equation}
\mu^- \st e^- + e^+ + e^-
\end{equation}

 В данной реакции не выполняется закон сохранения лептонных чисел. Несмотря на обозначение у, например, отрицательного мюона и электрона одинакового числа $L =  +1 $, эксперименты показывают, что должны выполнятся законы сохранения различных лептонных чисел для каждого поколения лептонов, и $ L_e \neq L_\mu $. 
 
 И хотя последние наблюдения показывают, что возможно нарушение данного правила для нейтрино (например, превращение электронного в мюонное), экспериментально не было обнаружено реакций с нарушением правила для остальных лептонов (таких, как в нашей реккции (8)). 

\section{Задача №7}

С помощью формул энергии и импульса релятивистской частицы найдем скорость электрона: 

\begin{equation}
p^2 = E^2/c^2 - m^2c^2 = \gamma^2 m^2v^2 
\end{equation}

где $ \gamma = \dfrac{1}{\sqrt{1-v^2/c^2}} $ --- релятивистский множитель. Решим это уравнение. % и для упрощения вычислений воспользуемся тем, что $ mc^2 \ll E $ (действительно, $ m_e \approx 0,5 \, \text{MeV}/c^2 $, а его энергия $ E = 50 \, \text{GeV} $). 

$$ \left( \dfrac{E^2}{c^2} - m^2c^2\right) \left(1 - \dfrac{v^2}{c^2}\right) = m^2v^2 ,$$
$$ \left( \dfrac{E^2}{c^2} - m^2c^2\right) - v^2\dfrac{E^2}{c^4} + v^2m^2   = m^2v^2,$$
$$  v^2 E^2 = E^2c^2 - m^2c^6   ,$$
\begin{equation}v = \sqrt{c^2 - \dfrac{m^2c^6}{E^2}} = c\sqrt{1 - \dfrac{m^2c^4}{E^2}} .
\end{equation}

Подставив значение $ m = 0,51 \, \text{MeV}/c^2 $ мы получаем  $ v = c\sqrt{1 - (\frac{0,51}{50000})^2} = \frac{99999999989596}{100000000000000}c$.

Тогда такой электрон пролетает $ s = 17 $ км за время $ \tau  = \dfrac{s}{v}  \approx 5,7 \x 10^{-5} $ c. Тогда за время $ T = 12 \text{ ч} = 43200 \text{ c} $ он совершит $ N = \dfrac{T}{\tau} \approx 7,6 \x 10^8 $ раз.

\begin{center}
	{\fbox{Ответ: $ N \approx 7,6 \x 10^8 $ раз}} \\
\end{center} 
%$$ E^4 - 2m^2E^2 + m^4 = \dfrac{m^2v^2}{1-v^2}, $$
%$$ (E^4 - 2m^2E^2)(1-v^2) = m^2v^2 ,$$
%$$ E^4 - 2m^2E^2 -E^4v^2 + 2m^2E^2v^2 = m^2v^2 ,$$
%$$ v^2(m^2 + E^4 - 2m^2E^2) = E^4 - 2m^2E^2, $$
%$$ v^2 (E^4 - 2m^2E^2) = E^4 - 2m^2E^2, $$




\end{document}