\documentclass[12pt]{urticle} 
\graphicspath{{../pictures/}}
\DeclareGraphicsExtensions{.pdf,.png,.jpg,.eps}

\begin{document}
	
\section{Задача 1.} Пусть система отсчета $ К $ двигается относительно системы $ К* $ со скоростью $ v_1. $
          А система $ К* $ двигается относительно системы $ К** $ со скоростью $ v_2 $.
          Предполагая , что координаты $ (х*,t*) $ с  $ (х,t) $, как и  $ (х**,t**) $ с  $ (х*,t*) $ связаны
          стандартными соотношениями Лоренца, проверить следует ли из этого, 
          что  $ (х**,t**) $ выражаются через  $ (х,t) $ аналогичными соотношениями , 
          в которых роль скорости системы отсчета $ К $ относительно  $ К** $ играет 
         $  u(v1,v2) $--функция от $ v1 $ и $ v2 $.
          
          Найти эту функцию.
          Утвердительный ответ для этой задачи означает, 
          что преобразования Лоренца образуют группу.


\section{Задача 2.} Используя вид уравнений Максвелла в терминах электрического $ Е_к $ и магнитного $ H_к $ полей $( к=1,2,3) $
          (то есть не переходя к векторному и скалярному потенциалам) , проверить , что, если функции координат 
           и времени $ Е_к $  и $ H_к $    удовлетворяют уравнениям Максвелла, то и $ Е*_к $  и $ H*_к $ --некоторые  определенные 
           линейными  комбинации полей $ Е_к $  и $ H_к $ ,  взятых в точке пространства-времени  с координатами $ (х*,y*,z*,t*) $,
           которые выражаются через $ (х,y,z,t) $ формулами преобразования Лоренца, для случая движения одной системы 
           отсчета относительно другой со скоростью в вдоль оси х.
           
           Найти явно коэффициенты в этих линейных комбинациях.
           Утвердительный ответ для этой задачи означает, 
           что уравнения Максвелла инвариантны относительно 
           преобразований Лоренца .

\section{Задача 3. }Пусть $ А(N) $ -алгебра Клиффорда с образующими $ g_n $, где $ n=1,...,N, $
          которые удовлетворяют соотношениям$  g_ng_m+g_mg_n=2d_{n,m} $,
          или $ {g_ng_m,g_mg_n}=2d_{n,m} $ где $ d_{n,m} $-символ Кронекера.
          В качестве базиса $ А(N) $ как векторного пространства можно взят $ e $-единицу в алгебре $ Е $,
          а также мономы $ g_{n_1}...g_{n_k} $  , где $ n_1<...<n_к<Н. $
          
          Найти число элементов этого базиса, то есть размерность алгебры $ А(N) $
          как векторного пространстваю 

\section{Задача 4.} Преставление алгебры $ А(N) $ для четного $ N $ матрицами $ 2^{N/2}х 2^{N/2} $ можно построить
          следующим образом. 
          Пусть V-векторное пространство размерности $ 2^{N/2} $. Поставим
          в   соответствие элементам алгебры $ А(N) $ линейные операторы, действующие в пространстве $ V $.
          Очевидно, что достаточно сделать это для генераторов $ g_n $.
          Удобно от $ g_n $ перейти к их линейным комбинациям  $ b_к^{+}  $и $ b_к^{-} $,где $ к=1,...,N/2. $

           $$  b_к^{+}=(g_{2к-1}+ig_{2к})/2 $$
           $$  b_к^{-}=(g_{2к-1}-ig_{2к})/2. $$

          Новые генераторы удовлетворяют соотношениям операторов "рождения" и уничтожения"

           $$  {b_к^{+},b_m^{+}}={b_к^{-},b_m^{-}}=0, $$
           $$   {b_к^{+},b_m^{-}}=d_{k,m} $$

          Поскольку операторы рождения $  b_к^{+} $ антикоммутируют между собой, как и операторы уничтожения $  b_к^{-} $,
          то в пространстве $ V  $существует вектор $ |v> $, который удовлетворяет условиям 

                $ b_к^{-} |v>=0 $.

          Действие на в операторов $ b_к^{+} $ определяет вектора 

                  $$   v(s_1,...,s_{N/2})=(b_1^{+})^s_1...(b_к^{+})s_{N/2} |v>, $$ где $  s_i=0,1 \forall i $.
             
           Покажите, что вектор $ |v> $ существует.
           Покажите, что вектора  $  v(s_1,...,s_{N/2}) $ линейно независимы, а их число равно $ 2^{N/2}. $
             
            Таким образом вектора $ v(s_1,...,s_{N/2} $ задают базис в пространстве $ V $.
            В этом базисе операторам $ b_к^{-} $ и $ b_к^{-} $, а значит и операторам  $ g_n $ 
            соответствуют матрицы $ 2^{Н/2}х2^{Н/2}. $
            Матрицы, соответствующие $ g_n $, называются матрицами Дирака,обозначим их $ G_n,  $
            определяются из соотношений

               $$  g_n  v(s_1,...,s_{N/2})= (G_n)_{s_1,...,s_{N/2}}^{t_1,...,t_{N/2}} v(t_1,...,t_{N/2}). $$

            kоэффициенты $ (G_n)_{s_1,...,s_{N/2}}^{t_1,...,t_{N/2}} $ являются матричными элементами матрицы $ G_n $.

            Найти матрицы Дирака $ G_n $ явно для случаев $ N=2 $ и $ N=4. $
                
                  
\end{document}