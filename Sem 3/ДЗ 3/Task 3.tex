\documentclass[12pt]{kiarticle} 
\graphicspath{{pictures/}}
\DeclareGraphicsExtensions{.pdf,.png,.jpg,.eps}
%%%
\pagestyle{fancy}
\fancyhf{}
%\renewcommand{\headrulewidth}{ 0.1mm }
\renewcommand{\footrulewidth}{ .0em }
\fancyfoot[C]{\texttt{\textemdash~\thepage~\textemdash}}
\fancyhead[L]{Некоторые концепции теоретической физики, ДЗ № 2\hfil}
\fancyhead[R]{\hfil Иванов Кирилл, 625 группа }

\newcommand{\Ll}{\ensuremath{\mathcal{L}}}
\newcommand{\deff}{\ensuremath{\stackrel{def}{=}}}


\begin{document}

\begin{titlepage}		
	\begin{center}
		\large 	Московский физико-технический институт \\
		Факультет общей и прикладной физики \\
		\vspace{0.2cm}
		Образовательная программа\\
		"<Квантовая теория поля, теория струн и математическая физика">
		
		\vspace{4.5cm}
		III семестр 2017-2018 учебного года \\ \vspace{0.1cm}
		\large Домашнее задание №2: \\ \vspace{0.1cm}
		\LARGE \textbf{Элементы классической теории поля, Лагранжев формализм}
	\end{center}
	\vspace{2.3cm} \large
	
	\begin{center}
		Автор: \\
		Иванов Кирилл,
		625 группа
		\vspace{10mm}
		
		
	\end{center}
	
	\begin{center} \vspace{50mm}
		г. Долгопрудный \\ 
		15 сентября 2017 года
	\end{center}
\end{titlepage}

%______________________________________________________________________________________________


%
%
%%%%%%%%%%%%%%%%%%%%%%%%%%%%%%%%%%%%%%%%
%
%

\section{Вопросы}

Определим скобку Пуассона для функций $ f(x_i), g(x_i) $, где $ x_i $ --- произвольные координаты на фазовом пространстве, $ i = 1, 2, \dots, 2n $:

\begin{equation}\label{}
\sak{f(x), g(x)} \deff \omega^{\mu\nu} \pdd{f}{x^\mu}\pdd{g}{x^\nu}
\end{equation}

где подразумевается суммирование по повторяющимся индексам (здесь и далее), а $ \omega^{\mu\nu} (x)$ --- антисимметричный тензор 2 ранга, т.е. $ \omega^{\mu\nu} (x) = -  \omega^{\nu\mu} (x)$.

Можно определить дифференциальную 2-форму $ \omega $ через внешнее произведение 1-форм $ dx^\mu $, и тогда будем говорить, что каждому нашему тензору $ \omega^{\mu\nu} $ соответствует форма $ \omega $, такая что

\begin{equation}\label{}
\omega = \omega^{\mu\nu} dx^\mu \wedge dx^\nu
\end{equation} 

Определим операцию внешнего дифференцирования формы как

 \begin{equation}\label{}
d = dx^\mu \pdd{}{x^\mu}
\end{equation}

Тогда форма $ A $ будет называться замкнутой, если $ dA = 0 $.






\end{document}