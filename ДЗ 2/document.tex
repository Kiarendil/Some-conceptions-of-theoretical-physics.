\documentclass[12pt]{article} 
\usepackage[left=20mm, top=16mm, right=16mm, bottom=20mm]{geometry} 
\usepackage{graphicx}
\usepackage{wrapfig}
\graphicspath{{../pictures/}}
\DeclareGraphicsExtensions{.pdf,.png,.jpg,.eps}
\usepackage{cmap}					% поиск в PDF
\usepackage{mathtext} 				% русские буквы в формулах
\usepackage[T2A]{fontenc}	
\usepackage[utf8x]{inputenc} 
\usepackage[russian]{babel} 
\usepackage{amsmath,amsfonts,amssymb,amsthm,mathtools} 
\usepackage{icomma} % "Умная" запятая: $0,2$ --- число, $0, 2$ --- перечисление
\usepackage{euscript}	 % Шрифт Евклид
\usepackage{mathrsfs} % Красивый матшрифт
%\usepackage{indentfirst}     % Отступ в первом абзаце
%% Перенос знаков в формулах (по Львовскому)
\newcommand*{\hm}[1]{#1\nobreak\discretionary{}
	{\hbox{$\mathsurround=0pt #1$}}{}}

%% Свои команды
%DeclareMathOperator{\sgn}{\mathop{sgn}}
\newcommand{\te}{\ensuremath{\Rightarrow}}
\newcommand{\y}{\ensuremath{\angle}}
\newcommand{\ABC}{\ensuremath{\triangle ABC\,}}
\newcommand{\tr}{\ensuremath{\triangle}}
\newcommand{\ca}{\ensuremath{\cos\alpha}}
\newcommand{\sa}{\ensuremath{\sin\alpha}}
\newcommand{\cb}{\ensuremath{\cos\beta}}
\newcommand{\sib}{\ensuremath{\sin\beta}}
\newcommand{\ov}{\ensuremath{\overline}}
\newcommand{\x}{\cdot}
\newcommand{\st}{\ensuremath{\longrightarrow}}
%DeclareMathOperator{\Sum}{\mathop{Sum}}
\DeclareMathOperator{\Sum}{\mathop{Sum}}

\begin{document}
	
\begin{titlepage}		
\begin{center}
\large 	Московский физико-технический университет \\
Факультет общей и прикладной физики \\
\vspace{0.2cm}
Учебная программа\\
"<Квантовая теория поля, теория струн и математическая физика">

\vspace{4.5cm}
II семестр 2016-2017 учебного года \\ \vspace{0.1cm}
\large Домашнее задание №2: \\ \vspace{0.1cm}
\LARGE \textbf{Взаимодействия элементарных частиц}
\end{center}
\vspace{2.3cm} \large

\begin{center}
		 Автор: \\
 Иванов Кирилл,
 625 группа
\vspace{10mm}


\end{center}

\begin{center} \vspace{50mm}
г. Долгопрудный \\ 
10 апреля 2017 года
\end{center}
\end{titlepage}

\section{Задача №1}
\begin{equation}
\begin{aligned}
\pi^- \st \mu^- + \overline{\nu} \\
\pi^+ \st \mu^+ + \nu
\end{aligned}
\end{equation}

Обозначив импульсы за $ p_{\mu}, p_{\pi}, p_{\nu} $, а энергии за  $ E_{\mu}, E_{\pi}, E_{\nu} $ соответственно, запишем законы сохранения энергии и импульса, принимая, что перед началом распада пион покоится $ \te p_{\pi} = 0 $:

\begin{equation}
\left\{
\begin{aligned}
&0 = \mathbf{p_{\mu} }+ \mathbf{p_{\nu}} \\
& E_\pi = E_\mu + E_\nu \\
&E_{\pi}^2 = m_\pi^2c^4 \\
&E_{\mu}^2 = m_\mu^2c^4 + p_\mu^2c^2 \\
&E_{\nu} =  p_\nu c \\
\end{aligned}
\right.
\te
\left\{
\begin{aligned}
&m_\pi c^2 = E_\mu + E_\nu \\
&E_{\mu}^2 = m_\mu^2c^4 + E_{\nu}^2 \\
\end{aligned}
\right.
\end{equation}

Решим эту простую систему:

$$ 
\left\{
\begin{aligned}
& (E_\mu + E_\nu)(E_\mu - E_\nu) = m_\mu^2c^4 \\
&E_\mu + E_\nu = m_\pi c^2  \\
\end{aligned}
\right.
\te 
\left\{
\begin{aligned}
& E_\mu - E_\nu = \dfrac{m_\mu^2c^2}{m_\pi} \\
&E_\mu + E_\nu = m_\pi c^2  \\
\end{aligned}
\right.
\te 
\left\{
\begin{aligned}
& E_\mu = \dfrac{c^2}{2m_\pi}\left( m^2_\mu+ m^2_\pi \right)  \\
& E_\nu = m_\pi c^2 - E_\mu \\
\end{aligned}
\right.
$$

Подставив значения $ m_\pi = 140 \text{ MeV}/c^2, m_\mu = 105 \text{ MeV}/c^2 $, находим, что 

$$ E_\mu = \dfrac{c^4}{2\x140}(140^2/c^4 + 105^2/c^4) \approx 109 \text{ MeV}/c^2, $$
$$ E_\nu \approx 140 - 109 = 31 \text{ MeV}/c^2 $$.

\begin{center}
	{\fbox{Ответ: $ E_\mu  \approx 109 \text{ MeV}/c^2, E_\nu \approx 31 \text{ MeV}/c^2  $}} \\
\end{center} 

\section{Задача 2}

\begin{equation}
\begin{aligned}
&a)\, \gamma + p \st p + e^- + e^+ \\
&b)\, \gamma + e^-  \st e^- + e^- + e^+ \\
\end{aligned}
\end{equation}

a) Известно, что энергия релятивистской частицы $E_r = \sqrt{m^2c^4 + p^2c^2} $. Для фотона $ \gamma $, у которого масса равна нулю, $ E = pc $. Обозначив массу покоящегося протона за $ M $, получаем, что инвариантная (т.е. та, которая сохраняется) энергия системы в начале равна $ E_{i1}^2 = (pc + Mc^2)^2 - (pc)^2 $, а затем, в реакции  получается покоящийся протон и 2 покоящихся электрона массой $ m $ (т.к. мы ищем минимальную энергию, берём покоящиеся частицы), инвариантная энергия которого есть энергия покоя системы: $ E_{i2}^2 = (Mc^2 + 2mc^2)^2 $. Для осуществления реакции необходимо $ E_{i1} > E_{i2} $:

\[ 
(E + Mc^2)^2 - E^2 > (Mc^2 + 2mc^2)^2 ,
\]
\[ 
E^2 + 2EMc^2 + M^2c^4  - E^2> M^2c^4 + 4Mmc^4 + 4m^2c^4 ,
\]
$$
EMc^2 > 2Mmc^4 + 2m^2c^4 ,
$$
$$
E > 2mc^2 \left(1 + \dfrac{m}{M}\right).
$$
Т.к. $ m = 0,5 \, \text{MeV}/c^2, M = 938 \text{MeV}/c^2 \te E > 1 \text{\,MeV}  $.
\par
б) Решаем аналогично:
$$ (E + mc^2)^2 - E^2 > (3mc^2)^2 , $$
$$	E^2 + m^2c^4 + 2Emc^2 - E^2 > 9m^2c^4	 $$
$$ 	2Emc^2 > 8m^2c^4 $$
$$ E > 4mc^2 \te  E > 2 \text{\,Mev} $$.

\begin{center}
{\fbox{Ответ: а) $ E_{min} = 1 \text{\,MeV}  $ б) $E_{min} =  2 \text{\,Mev}$}} \\
\end{center} 

\section{Задача 3}

\begin{equation}
\gamma + p \st p + \mu^+ + \mu^- 
\end{equation} 

Обозначив массу покоящегося протона за $ M $, массу мю-мезона за $m$ и энергию фотона за $E$, решаем такую же задачу:  

$$ (E + Mc^2)^2 - E^2 > (Mc^2 + 2mc^2)^2 , $$
$$ E > 2mc^2 \left(1 + \dfrac{m}{M}\right). $$

Т.к. $ m = 105 \, \text{MeV}/c^2, M = 938 \text{MeV}/c^2 \te E > 233,5 \text{\,MeV}  $.

\begin{center}
	{\fbox{Ответ: $ E_{min} = 233,5 \text{\,MeV} $}} \\
\end{center} 

\section{Задача 4}

Лептонное число электрона и мюона, а также соответствующих им нейтрино $L^+(e^-, \nu_{e}, \mu^-, \nu_{\mu}) \hm{=} +1$, а число их античастиц $L^-(e^+, \overline{\nu}_e, \mu^+, \overline{\nu}_\mu) \hm{=} -1$. Число нелептонов (таких как $ p, n, \gamma $ и их античастиц) равно $0$. \vspace{0.3cm}

a) $ \mu^- + p \st e^- + p $:\\ реакция не идет, так как не выполняется \textbf{закон сохранения нуклонного (зарядового) числа} (слева +1, справа 0).

b) $ p \st e^+ + \gamma $:\\ реакция {не идет, так как не выполняется \textbf{закон сохранения лептонного числа} (слева 0, справа $ -1 $).
	
c) $ \gamma + p \st n + e^+ $:\\ реакция не идет, так как не выполняется \textbf{закон сохранения лептонного числа} (слева 0, справа $ -1 $).

d) $ n + p \st \overline{p} + \ov{n} + e^+ + e^+ $:\\ реакция не идет, так как не выполняется\textbf{ закон сохранения лептонного числа} (слева 0, справа -$ 2 $).

e) $ e^- + p \st \ov{\nu} + n $:\\ реакция не идет, так как не выполняется \textbf{закон сохранения нуклонного (зарядового) числа} (слева $ -1 $, справа 0).

f) $ p + p \st p + p + p + \ov{p} $:\\ здесь законы сохранения обоих чисел выполнены, т.е. такая реакция возможна.




\end{document}